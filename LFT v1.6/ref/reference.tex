% Reference

\begin{document}
参考文献 

[1]  娄素华,卢斯煜,吴耀武,尹项根.低碳电力系统规划与运行优化研究综述[J].
     电网技术,2013,37(6):1483-1490.

[2]  王鸣晓,林建民,马光进,周淑玲,董涛.电气设备防雷工程设计探讨[J].
     气象科技,2013,41(2):417-421.

[3]  赵洪涛.电力系统规划分类及方法探究[J].
     中国电力教育,2013,11:184-185.

[4]  关蕾.绥中电厂2×1000MW超超临界机组工程造价估算方法研究[D].
     华北电力大学(北京),2010.

[5]  张慧敏.试论天津地区燃煤电厂的大气污染及其治理对策[J].
     天津电力技术,2000,3:19-21. 

[6]  周小谦.我国“西电东送”的发展历史、规划和实施[J].
     电网技术,2003,27(5):1-5.

[7]  陈跃.电力工程专业毕业设计指南(电力系统分册) [M].
     北京:中国水利水电出版社,2003.

[8]  Lin Weixing,Wen Jinyu,Hu Jianglu,et al.An investigation on the active-power variations of wind farms[J].
     IEEE Transactions on Industry Applications,2012,

[9]  Lin Weixing,Wen Jinyu,Hu Jianglu,et al.An investigation on the active-power variations of wind farms[J].
     IEEE Transactions on Industry Applications,2012,

     【说明】

     1.参考文献也需另起一页,不按章编号。

     2.行距为1.5倍行距。中文为小四宋体,英文为小四Times New Roman,
     编号后空2小格.悬挂缩进与首行对齐.标点除括弧为半角外,其余均为全角.
     
     3.参考文献必须在12篇以上,英文文献至少1篇,近5年文献至少6篇,
     其中文献应以科技类期刊杂志中的科技论文为主,本科学习阶段的教材尽量少列为参考文献。
     参考文献要全面,比如:“电力系统规划及发电厂电气部分设计”尽量涉及国家能源政策、
     系统规划、工程造价、环境影响、防雷等方面的文献。
     

以下是参考文献截图,展示规范文档要求
%图片,后附


\end{document}