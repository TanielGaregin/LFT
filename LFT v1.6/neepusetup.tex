%插入使用的包体并描述一些格式定义设置

%论文信息,文章格式要求存在但是这里并没有用
\title{应该不会搞乱吧}
\author{}
\date{}

%适用全文的格式设置
\usepackage{ctex}%中文输入
\usepackage[top=2.54cm, bottom=2.54cm, left=3.17cm, right=3.17cm]{geometry} %页边距
\usepackage{fontspec}%全文改字体
\usepackage{setspace}%行间距
\usepackage{changepage}%判断页面奇数或者偶数、缩进
\usepackage{ragged2e}%断词,左右对齐
\usepackage{amsmath,amsfonts}%公式定理排版+数学符号
%\setlength{\textwidth}{14.66cm}%定义文本可用宽度14.66cm=415.56pt=5.77165in
\setmainfont{TeX Gyre Termes}%用一个类似Times New Roam字体的字体 
\linespread{1.5}%设置1.5倍行距
\usepackage{url}%文中超链接,脚注
\usepackage{verbatim}%加入评论和直接抄录文件

%页眉
\usepackage{fancyhdr}%页眉页脚
\pagestyle{fancy}
\fancyhf{}
\fancyhead[C]{\zihao{-5} 东北电力大学本科毕业设计论文}    %页眉设置
\fancyfoot[C]{\thepage}           %页脚处显示页码
\renewcommand{\headrulewidth}{0.5pt}    %页眉线


%摘要格式设置
\newcommand{\enabstractname}{{\includegraphics[scale=0.06]{ABSTRACT-sun.pdf}}}
\newcommand{\cnabstractname}{\LARGE{摘 \qquad 要}}
\newenvironment{enabstract}{%
  \par
  \noindent\mbox{}\hfill{\bfseries \enabstractname}\hfill\mbox{}\par
  \vskip 2.5ex}{\par\vskip 2.5ex}
\newenvironment{cnabstract}{%
  \par
  \noindent\mbox{}\hfill{\bf \cnabstractname}\hfill\mbox{}\par
  \vskip 2.5ex}{\par\vskip 2.5ex}

%目录格式设置
\usepackage[newparttoc,indentafter]{titlesec}%章节标题
\usepackage{titletoc}%目录排版形式
\renewcommand\contentsname{目\hspace{1em}录}
% \titlecontents{章节名称}[左端距离]{标题字体、与上文间距等}{标题序号}{空}{引导符和页码}[与下文间距]
\titlecontents{section*}[]{\heiti \zihao{-4}}{}{}{}
\titlecontents{section}[0em]{\heiti \zihao{-4} }{第 \thecontentslabel 章 ~~ }{}{\hspace{0.5em}\titlerule*[0.4pc]{$.$}\hspace{-0.5em}\contentspage}[\vspace{0.3em}]
\titlecontents{subsection}[2em]{\heiti \zihao{-4} }{\contentslabel{2em}}{}{\hspace{0.5em}\titlerule*[0.4pc]{$.$}\hspace{-1em}\contentspage}[\vspace{0.3em}]
\titlecontents{subsubsection}[4em]{\songti \zihao{-4} }{\contentslabel{3em}}{}{\hspace{0.5em}\titlerule*[0.4pc]{$.$}\hspace{-1em}\contentspage}[\vspace{0.3em}]
\usepackage{hyperref} %交叉引用,目录超链接

%各级标题格式设置
\usepackage{bookmark}%添加无编号章节
\setcounter{secnumdepth}{4}
\titleformat{\section}[block]{\LARGE \heiti \centering }{第\arabic{section}章}{1em}{\vspace{0.5\baselineskip}}[]
\titleformat{\subsection}[block]{\zihao{3} \heiti}{\arabic{section}.\arabic{subsection}}{1em}{}[]
\titleformat{\subsubsection}[block]{\zihao{4} \textbf \songti}{\arabic{section}.\arabic{subsection}.\arabic{subsubsection}}{1em}{\textbf}[]

%图片位置及多图排版
\usepackage{graphicx} %插入图片的宏包
\usepackage{float} %设置图片浮动位置的宏包


%图表标题设置
\usepackage{caption}%提供了多种命令来更方便的设计浮动图形和表格的标题式样。
%\usepackage{caption2}%另一个功能强大的设计浮动对象的标题式样的宏包
\usepackage{subcaption}%子图表标题
\captionsetup{font={small}}%图标标题字号
\renewcommand{\thefigure}{\thesection-\arabic{figure}}
\renewcommand{\thetable}{\arabic{section}-\arabic{table}}   % 设置表格编号样式

%表格
\usepackage{array}%tabular增强
\usepackage{ltxtable}% longtable 和 tabularx 两个宏包的结合
\usepackage{multirow}%合并单元格
\usepackage{textcomp}%提供特殊单位符号
\usepackage{stfloats}%自定义图片的位置,[htbp]
\usepackage{booktabs}%三线表不同线粗
\usepackage{threeparttable}%三线表可用
\usepackage{longtable}%跨页表
\usepackage{makecell}%表格内换行

%表格占满行设置的一种方法
\usepackage{tabularx} % for 'tabularx' env. and 'X' col. type
%\usepackage{ragged2e} % for \RaggedRight macro
\newcolumntype{L}{>{\RaggedRight\hangafter=1\hangindent=0em}X}
\newcolumntype{C}{>{\Centering \hangafter=1\hangindent=0em}X}
\newcolumntype{R}{>{\RaggedLeft \hangafter=1\hangindent=0em}X}


%公式右编号设置
\renewcommand{\theequation}{\arabic{section}-\arabic{equation}}  % 设置公式编号样式

%表格注释
\makeatletter
\newsavebox{\@tabnotebox}
\providecommand\tmark{} % so having ctable or not is irrelevant
\providecommand\tnote{}
\newenvironment{tabularwithnotes}[3][c]
  {\long\def\@tabnotes{#3}%
  \renewcommand\tmark[1][a]{\makebox[0pt][l]{\textsuperscript{\itshape##1}}}%
  \renewcommand\tnote[2][a]{\textsuperscript{\itshape##1}\,##2\par}
  \begin{lrbox}{\@tabnotebox}
   \begin{tabular}{#2}}
  {\end{tabular}\end{lrbox}%
  \parbox{\wd\@tabnotebox}{
    \usebox{\@tabnotebox}\par
    \smallskip\@tabnotes
   }%
  }
\makeatother

%算法排版
\usepackage{algorithmicx}%算法排版环境
\usepackage{algorithm}%排版算法步骤的 algorithmc 和 algorithm 环境

%参考文献和引用
%\usepackage{cite}%引用环境
\usepackage[numbers,sort&compress]{natbib}%设置引用环境,数字形式引用并合并引用序号
\usepackage[title,titletoc]{appendix}%附录宏包
\usepackage{tocbibind}%参考文献或索引、目录等放置到目录中
\usepackage{bibentry}%文本的任何地方放置参考文献的条目
\newcommand{\upcite}[1]{\textsuperscript{\cite{#1}}}%上标引用格式定义
\usepackage{gbt7714}
\bibliographystyle{gbt7714-numerical}
\renewcommand\refname{参考文献}   %改参考文献标题为中文
\renewcommand{\bibname}{参 \hspace{0.075em} 考 \hspace{0.075em} 文 \hspace{0.075em} 献}