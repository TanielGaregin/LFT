% 中文摘要

\begin{cnabstract}
    \addcontentsline{toc}{section}{摘 \hspace{0.45em} 要} %手动添加为目录
     \thispagestyle{fancy}%页眉显示
     %\fancyhf{}
     随着电力电子器件的发展,PWM电压型逆变器在交流变频调速、UPS、电能质量控制器、轻型直流输电换流器等
     电力电子装置中得到了越来越广泛的应用。PWM电压型逆变器直流侧所需的理想无脉动直流电压源通常通过整流
     加上大直流电容滤波获得。大直流滤波电容的使用,给装置带来占用空间大、成本高及严重影响电能质量方面的
     问题。因此,研究如何减小甚至去除逆变器直流侧电容,以及解决因其产生的低次谐波和相关问题,

     具有十分重要的理论意义和实用价值。本文在综述了国内外在PWM电压型逆变器及各种抑制谐波PWM技术的基础上,
     对目前工程中应用最广泛的SPWM电压型逆变器的主电路及谐波消除调制技术和相关问题进行了深入研究。
    \par
    \vspace{1\baselineskip}
    \noindent\textbf{关键字: } \parbox[t]{30.8em}{PWM逆变器;直流电容;调制波重构SPWM;六脉动直流电压;PWM逆变器;直流电容;调制波重构SPWM;六脉动直流电压;PWM逆变器;直流电容;调制波重构SPWM;六脉动直流电压 }
    %关键词的另一种排版形式如下注释部分    
    %\begin{minipage}[t]{0.85\linewidth}
    %PWM逆变器;直流电容;调制波重构SPWM;六脉动直流电压            
    %\end{minipage}
\end{cnabstract}
    \pagenumbering{Roman} %页码为大写罗马数字
    
\newpage
%    \addcontentsline{toc}{section}{ABSTRACT}
%     \thispagestyle{fancy}
%    \begin{enabstract}
        %\centering
        %\includegraphics[scale=0.3]{ABSTRACT.png}\\
        %\includegraphics[scale=1.1]{ABSTRACT-sun.pdf}\\
        %{ \LARGE \textbf{ABSTRACT}}\\
%        \begin{spacing}{1.5}
%            \justify
%          \qquad With the rapid development of the power and electron parts, PWM voltage inverter is 
%        applied, on a larger and larger scale, to ac frequency control system 、UPS、power 
%        quality controller and Converter of high light HVDC transmission. Non-ripple DC voltage 
%        source of PWM voltage type Inverter is gained by rectifiers paralleled connection a large 
%        DC filter capacitor. The equipment that uses a large DC filter capacitor has disadvantages, 
%        such as bulk, high price and low power quality. It is therefore becoming most significant 
%        and valuable to do research on how to reduce capacitance as small as possible or even 
%        remove capacitor and how to settle relative issues and to eliminate low order harmonics 
%        due to minishing capacitance or removing capacitor. Based on the survey of main circuits 
%        of inverter and restraining harmonic effects of PWM techniques, the main circuits and 
%        harmonic elimination PWM techniques of SPWM voltage type inverter, which applied widely 
%        in industry fields, are profoundly analyzed.
        
%        \end{spacing}
%        \vspace{1\baselineskip}
%    \noindent \textbf{Keywords:}
%    \begin{minipage}[t]{0.85\linewidth}
%        PWM inverter; DC capacitor; Six-ripple DC voltage; Modulation wave reconstruction SPWM
%    \end{minipage}
%\end{enabstract}


%\newpage
%    【中文摘要及关键词说明】

%  1.摘要又称内容提要,它应以浓缩的形式概括研究课题的内容、方法和观点,以及取得的成果和结论,
%  应能反映整个内容的精华。中外文摘要以300-500字为宜,要独立成文,选词用语要避免与全文
%  尤其是前言和结论部分雷同。
%  2.“摘要”标题二字为小二号、黑体不加粗、居中排版,两字中间空2格;段前设置0.5行;段后设置1行。
%  3.正文缩进二字书写,采用小四号、宋体、不加粗形式,行间距为1.5倍行距。
%  4.摘要内容后空一行顶格打印“关键词:”(小四号黑体字),其后为各关键词(小四号宋体字),
%  各关键词之间用逗号分开,最后一个关键词后面无标点符号。关键词一般以3—5个为宜。


%    【英文摘要说明】
%    1.书写格式与中文摘要对应,英文均采用“Times New Roman”字体,小四不加粗,行间距1.5倍行距。
%    2.“摘要”二字英文作为标题一律用大写字母,小二号宋体加粗。
%    3.排版时正文首行缩进两个汉字的字符,两端对齐。
%    4.英文“关键词:”翻译形式为“Keywords:”首个字母大写,其余小写,小四字体加粗。各关键词英文翻译
%    之间用分号隔开,最后一个后面不加符号。换行后首个字母与第一个关键词的首字母对齐,而非顶格书写,小四号字。
%    5.摘要正文后空一行开始撰写关键词。各关键词除专业术语等按语法规则该大写的以外,其余全部用小写字母,包括首字母。

%\newpage

%目录页
%\tableofcontents
\clearpage