% 英文摘要及目录
    
\newpage
    \addcontentsline{toc}{section}{ABSTRACT}
     \thispagestyle{fancy}
    \begin{enabstract}
        %以下注释部分代码作用是加入宋体形式英文摘要标题,如果失效,可以适当启用ABSTRACT-sun.pdf作为标题
        %\centering
        %\includegraphics[scale=0.3]{ABSTRACT.png}\\
        %\includegraphics[scale=1.1]{ABSTRACT-sun.pdf}\\
        %{ \LARGE \textbf{ABSTRACT}}\\
        \begin{spacing}{1.5}
            \justify
          \qquad With the rapid development of the power and electron parts, PWM voltage inverter is 
        applied, on a larger and larger scale, to ac frequency control system 、UPS、power 
        quality controller and Converter of high light HVDC transmission. Non-ripple DC voltage 
        source of PWM voltage type Inverter is gained by rectifiers paralleled connection a large 
        DC filter capacitor. The equipment that uses a large DC filter capacitor has disadvantages, 
        such as bulk, high price and low power quality. It is therefore becoming most significant 
        and valuable to do research on how to reduce capacitance as small as possible or even 
        remove capacitor and how to settle relative issues and to eliminate low order harmonics 
        due to minishing capacitance or removing capacitor. Based on the survey of main circuits 
        of inverter and restraining harmonic effects of PWM techniques, the main circuits and 
        harmonic elimination PWM techniques of SPWM voltage type inverter, which applied widely 
        in industry fields, are profoundly analyzed.
        
        \end{spacing}
        \vspace{1\baselineskip}
    \noindent \textbf{Keywords:}
    \parbox[t]{29.9em}{PWM inverter; DC capacitor; Six-ripple DC voltage; Modulation wave reconstruction SPWM}
    %以下是关键词部分的另一种设置形式
    %\begin{minipage}[t]{0.85\linewidth}
        %PWM inverter; DC capacitor; Six-ripple DC voltage; Modulation wave reconstruction SPWM
    %\end{minipage}
\end{enabstract}

%    【英文摘要说明】
%    1.书写格式与中文摘要对应,英文均采用“Times New Roman”字体,小四不加粗,行间距1.5倍行距。
%    2.“摘要”二字英文作为标题一律用大写字母,小二号宋体加粗。
%    3.排版时正文首行缩进两个汉字的字符,两端对齐。
%    4.英文“关键词:”翻译形式为“Keywords:”首个字母大写,其余小写,小四字体加粗。各关键词英文翻译
%    之间用分号隔开,最后一个后面不加符号。换行后首个字母与第一个关键词的首字母对齐,而非顶格书写,小四号字。
%    5.摘要正文后空一行开始撰写关键词。各关键词除专业术语等按语法规则该大写的以外,其余全部用小写字母,包括首字母。

%目录页
\newpage
\tableofcontents
\clearpage