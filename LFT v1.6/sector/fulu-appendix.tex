
\begin{appendices}

    \section*{附录1  }
    \addcontentsline{toc}{section}{1}
    \subsection*{【封面说明】}
    1.大论文标题为一号黑体字不加粗,需换行时应该在一个完整词义结束后强制换行。

    2.标题字数要适当,一般不宜超过20字,如果有些细节必须放进标题,
    为避免冗长,可以分成主标题和副标题,主标题写得简明,将细节放在副标题里。

    3.特别提醒:封面上不能出现页眉中的横线!

    4.封面表格日期根据实际答辩日期填写

    \subsection*{【中文摘要及关键词说明】}
    1.摘要又称内容提要,它应以浓缩的形式概括研究课题的内容、方法和观点,
    以及取得的成果和结论,应能反映整个内容的精华。中外文摘要以300-500字为宜,
    要独立成文,选词用语要避免与全文尤其是前言和结论部分雷同。

    2.“摘要”标题二字为小二号、黑体不加粗、居中排版,两字中间空2格;
    段前设置0.5行;段后设置1行。

    3.正文缩进二字书写,采用小四号、宋体、不加粗形式,行间距为1.5倍行距。

    4.摘要内容后空一行顶格打印“关键词:”(小四号黑体字),其后为各关键词
    (小四号宋体字),各关键词之间用逗号分开,最后一个关键词后面无标点符号。
    关键词一般以3—5个为宜。

    \subsection*{【关于页眉、页脚的说明】}
    1.页眉设置采用:单线0.5磅;上面用小五号宋体不加粗书写“东北电力大学本科毕业设计论文”,
    文字居中排版,奇偶页页眉一致。

    2.页脚设置要求:中英文摘要、目录要求单独编排页脚编号,采用“\uppercase\expandafter{\romannumeral1}、\uppercase\expandafter{\romannumeral2}、\uppercase\expandafter{\romannumeral3}、\uppercase\expandafter{\romannumeral4}、……”
    编号形式;第1章绪论开始采用“1、2、3、……”数字编号形式,直至最后一页统一排序。
    数字编号均采用小五号宋体不加粗形式。

    \subsection*{【英文摘要说明】}
    1.书写格式与中文摘要对应,英文均采用“Times New Roman”字体,小四不加粗,行间距1.5倍行距。

    2.“摘要”二字英文作为标题一律用大写字母,小二号宋体加粗。

    3.排版时正文首行缩进两个汉字的字符,两端对齐。

    4.英文“关键词:”翻译形式为“Keywords:”首个字母大写,其余小写,
    小四字体加粗。各关键词英文翻译之间用分号隔开,最后一个后面不加符号。
    换行后首个字母与第一个关键词的首字母对齐,而非顶格书写,小四号字。

    5.摘要正文后空一行开始撰写关键词。各关键词除专业术语等按语法规则该大写的以外,
    其余全部用小写字母,包括首字母。

    \subsection*{【目录说明】}
    1.标题“目录”二字中间空两个格,其余格式与中文摘要标题相同。

    2.目录要求列出三级标题,第一级和第二级标题顶格书写,采用小四黑体不加粗形式;
    第三级标题缩进二个汉字字符书写,采用小四宋体不加粗形式;行距采用1.5倍。

    3.(word)目录最好采用“自动生成的目录”的方法,具体步骤见下面提示。
    (word)【目录自动生成方法】
    在目录处点击鼠标右键,选择“更新域”,弹出对话框后,选择“更新整个目录”,即可完成目录的自动生成。

    \subsection*{【正文章节说明】}

    \subsubsection*{章节标题}
    (小二黑体不加粗,居中,章号与题名之间空两格,标题段前0.5行,段后1行,行距均为1.5倍行距。)

    \subsubsection*{二级标题}
    (三号黑体不加粗,左对齐,标号与题目之间空两格,标题段前0.5行,段后1行)

    \subsubsection*{三级标题}
    (四号宋体加粗,左对齐,标号与题目之间空两格,标题段前0.5行,段后1行。)

    \subsubsection*{正文内容}
    (四号宋体,两端对齐,1.5倍行距)

    \subsubsection*{【大论文正文书写要求】}
    1.正文中任何部分不得打印到页边距之外,稿纸不得随意接长或截短。

    2.正文字数不少于1.5-2万字含设计分析与计算、实验及数据处理、程序等;
    正文内容为小四号字体不加粗,1.5倍行距,首行缩进二个字;

    3.公式应另起一行写在稿纸中央,一行写不完的长公式,最好在等号处转行,
    如做不到这点,在数学符号(如“+”、“-”号)处转行,数学符号应写在转行后的行首。
    公式的编号用圆括号括起放在公式右边行末,在公式和编号之间不加虚线;
    公式序号按章节顺序编号;重复引用的公式不得另编新序号。示例见2.1.1。

    4.表格按章节顺序编号,表号、表题应放在表格上方,
    表号、表题及表中用到文字应较正文小一号,采用五号宋体;表格允许下页接写,
    接写时表题省略,表头应重复书写,并在右上方写“续表××”。
    此外,表格应写在离正文首次出现处的近处,不应过分超前或拖后。示例见2.1.1。

    5.毕业设计(论文)的插图必须精心制作,线条要匀称,图面要整洁美观,
    插图应与正文呼应,不得与正文脱节。每幅插图应有图号和图题,按章节顺序编号,
    图中坐标应标注单位,图号、图题应放在图的下方,
    图号、图题及图中涉及的文字字号应较正文小一号,采用五号宋体。由若干分图组成的插图,
    分图用a,b,c…标序,分图的图名以及图中各种代号的意义,以图注形式写在图题下方,
    先写分图名,另起行后写代号的意义。图应在描纸或洁白纸上用墨线绘成,或用计算机绘图,
    应符合相应国家标准的要求。示例见2.1.1。

    6.毕业设计(论文)中有个别名词或情况需要解释时,可加注说明,
    注释可用页末注(将注文放在加注页稿纸的下端)或篇末注(将全部注文集中在文章末尾),
    而不用行中注(夹在正文中的注)。若在同一页中有两个以上的注时,按各注出现的先后,
    须序编列注号,注释只限于写在注释符号出现的同页,不得隔页。
    (公式中涉及的字母所代表的物理量要全部指明。右对齐,
    符号大小在公式编辑器中尺寸栏中的“define”下的“full”项选12,其他默认)
    正文中的图。(图题在图的下方,居中,图题文字为五号宋体,图中文字为小五号)
    正文中的表。(采用三线表,表题为五号宋体,表中为小五宋体。表题在表上方,居中)

    7.绪论中末小节必须为“本文完成的主要工作”
    
    \subsection*{【结论 说明】}
    1.结论要单起一页,不按章编号,“结论”二字中间空两格。

    2.一般在500-800字为宜。

    3.结论应反映个人的研究工作,属于他人已有过的结论要少提。

    \subsection*{【参考文献 说明】}
    1.参考文献也需另起一页,不按章编号。

    2.行距为1.5倍行距。中文为小四宋体,英文为小四Times New Roman,编号后空2小格.
    悬挂缩进与首行对齐.标点除括弧为半角外,其余均为全角.

    3.参考文献必须在12篇以上,英文文献至少1篇,近5年文献至少6篇,
    其中文献应以科技类期刊杂志中的科技论文为主,本科学习阶段的教材尽量少列为参考文献。
    参考文献要全面,比如:“电力系统规划及发电厂电气部分设计”尽量涉及国家能源政策、
    系统规划、工程造价、环境影响、防雷等方面的文献。

    \subsection*{【致谢 说明】}
    致谢应以简短的文字对课题研究与论文撰写过程中曾直接给予帮助的人员
    (例如指导教师、答疑教师及其他人员)表示自已的谢意,这不仅是一种礼貌,
    也是对他人劳动的尊重,是治学者应有的思想作风。
    \subsection*{【附录 说明】}
    对于一些不宜放入正文中、但作为毕业设计(论文)又不可残缺,的组成部分,
    或有主要参考价值的内容,可编入毕设计(论文)的附录中,例如,公式的推演、
    编写的算法语言程序等。如果毕业设计中引用的实例、数据资料,实验结果等符号较多时,
    为了节约篇幅,便于读者查阅,可以编写一个符号说明,注明符号代表的意义。
    附录的篇幅不宜太多,附录一般不要超过正文。


    \section*{附录2  }
    \addcontentsline{toc}{section}{2}
    some text in Appendix B


\end{appendices}
