\begin{center}

    \begin{figure}[h]
        \centering
        \vspace{2.6\baselineskip}
        \includegraphics[width=9.49cm ]{neepu2024.png}
        %\includegraphics[width=6.8cm ]{neepu.png}
        %\includegraphics[width=6.8cm ]{NEEPU-sign.png}
    \end{figure}
    \vspace{0.27em}    
    {\scalebox{0.96}[1]{\fontsize{38pt}{0}\heiti \hspace{-5pt}毕 \hspace{5pt} 业 \hspace{5pt} 设 \hspace{5pt} 计 \hspace{5pt} 论 \hspace{5pt} 文\hspace{5pt} }}  \\  %38磅黑体
    \vspace{2.9em}

    {\zihao{1}\heiti
    
    直流电压源逆变器及其消谐调制\\[0.8em] 技术研究    
    }%手动分行,\\[0.8em]分隔

\end{center}
\thispagestyle{empty}
\vspace{7.2\baselineskip}

%【封面说明】
%1.大论文标题为一号黑体字不加粗,需换行时应该在一个完整词义结束后强制换行。
%2.标题字数要适当,一般不宜超过20字,如果有些细节必须放进标题,为避免冗长,可以分成主标题和副标题,主标题写得简明,将细节放在副标题里。
%3.特别提醒:封面上不能出现页眉中的横线!
%4.根据实际答辩日期填写




\begin{table}[htbp]
    \centering
    \linespread{1.8}
    \zihao{4}
    \heiti
    \renewcommand\arraystretch{1.028}
\begin{tabular}{p{4.58cm}<{\raggedleft} p{6.88cm}<{\centering}}
学\hspace{0.54em} 生\hspace{0.54em} 姓\hspace{0.54em} 名\ :&呼延汀  \\%[0.6cm]
班 \hspace{3.9em} 级\ :                                   &电自9532 \\%[0.6cm]
学 \hspace{3.9em} 号\ :                                   &1095703063214  \\%[0.6cm]
指\hspace{0.54em} 导\hspace{0.54em} 教\hspace{0.54em} 师\ :&公孙玲珑  \\%[0.6cm]
所\hspace{0.54em} 在\hspace{0.54em} 单\hspace{0.54em} 位\ :&\  \  \  \  电气工程学院  \\%[0.6cm]
答\hspace{0.54em} 辩\hspace{0.54em} 日\hspace{0.54em} 期\ :&\  \  1099年\  \  06月\  \  11日    \\%[0.6cm]

\end{tabular}
\end{table}
\cleardoublepage