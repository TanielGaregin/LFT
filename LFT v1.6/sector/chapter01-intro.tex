% Introduction
\clearpage
\pagenumbering{arabic} %页码为阿拉伯数字
    \section{绪 \hspace{0.5em} 论}
    %(小二黑体不加粗,居中,章号与题名之间空两格,标题段前0.5行,段后1行,行距均为1.5倍行距。)

    \subsection{课题背景及研究的意义}
    %(三号黑体不加粗,左对齐,标号与题目之间空两格,标题段前0.5行,段后1行)

    \subsubsection{课题背景}
    %(四号宋体加粗,左对齐,标号与题目之间空两格,标题段前0.5行,段后1行。)

    50年代末晶闸管标志着电力电子半导体期间的开端。电力电子器件经历了40多年的
    发展历程[1,2],特别是近30多年内更是得到了迅猛的发展[3,4]。以Th(SCR)为代表的半控型器件是
    第一代电力电子器件[5],其主要用于可控整流装置,若用于可控的逆变器,因其无法自行关断,
    须配置强迫换流电路,致使装置复杂化。70年代中期,相继研制成功的电流控制型的双极性晶体管
    (Bipolar Junction Transistor——BJT)、门极可关断晶闸管(Gate Turn-off Thyristor——GTO)
    以及电压控制型的电力场效应晶体管(Power MOS Field-effect Transistor——P-MOSFET)等
    全控型器件被称为第二代电力电子器件[6]%(四号宋体,两端对齐,1.5倍行距)

    \subsubsection{课题研究的意义}

    由于PWM逆变器的广泛应用及谐波会产生上述的诸多危害,因此必须对PWM逆变器的主电路
    及其谐波抑制技术进行研究。

    \subsection{PWM逆变器研究现状}

    所谓逆变器,是指整流(又称顺变)器的逆向变换装置。作为现代电力电子技术中最基本装置之一的
    PWM电压型逆变器是随着器件和控制技术的发展而不断发展起来的,采用PWM逆变技术的目的是为了
    获得不同或变化形式的电能。早期的半导体器件是普通的晶闸管半控型器件,其开关频率很低,
    逆变输出的交流电压的波形基本上是方波型。

    \subsection{本文完成的主要工作}%【必有内容】

    综上所述,为了进一步提高应用最为广泛的SPWM电压型逆变器的性能,获得良好的经济和社会效益,
    必须对其主电路及谐波消除调制技术进行研究,解决SPWM逆变器主电路及谐波等问题。

    本文所要完成的主要内容包括以下几个方面:

    (1)	\quad \space 传统SPWM电压型逆变器为减……

    (2)	\quad \space 传统的SPWM……

    (3)	\quad \space 对本文所提……

    (4)	\quad \space 本文运用……



    
    %【大论文正文书写要求】

    %1.正文中任何部分不得打印到页边距之外,稿纸不得随意接长或截短。

    %2.正文字数不少于1.5-2万字含设计分析与计算、实验及数据处理、程序等;
    %正文内容为小四号字体不加粗,1.5倍行距,首行缩进二个字;

    %3.公式应另起一行写在稿纸中央,一行写不完的长公式,最好在等号处转行,
    %如做不到这点,在数学符号(如“+”、“-”号)处转行,数学符号应写在转行后的行首。
    %公式的编号用圆括号括起放在公式右边行末,在公式和编号之间不加虚线;公式序号按章节顺序编号;
    %重复引用的公式不得另编新序号。示例见2.1.1。

    %4.表格按章节顺序编号,表号、表题应放在表格上方,表号、表题及表中用到文字应较正文小一号,
    %采用五号宋体;表格允许下页接写,接写时表题省略,表头应重复书写,并在右上方写“续表××”。
    %此外,表格应写在离正文首次出现处的近处,不应过分超前或拖后。示例见2.1.1。

    %5.毕业设计(论文)的插图必须精心制作,线条要匀称,图面要整洁美观,插图应与正文呼应,不得与正文脱节。
    %每幅插图应有图号和图题,按章节顺序编号,图中坐标应标注单位,图号、图题应放在图的下方,
    %图号、图题及图中涉及的文字字号应较正文小一号,采用五号宋体。由若干分图组成的插图,分图用a,b,c…标序,
    %分图的图名以及图中各种代号的意义,以图注形式写在图题下方,先写分图名,另起行后写代号的意义。
    %图应在描纸或洁白纸上用墨线绘成,或用计算机绘图,应符合相应国家标准的要求。示例见2.1.1。

    %6.毕业设计(论文)中有个别名词或情况需要解释时,可加注说明,注释可用页末注(将注文放在加注页稿纸的下端)
    %或篇末注(将全部注文集中在文章末尾),而不用行中注(夹在正文中的注)。若在同一页中有两个以上的注时,
    %按各注出现的先后,须序编列注号,注释只限于写在注释符号出现的同页,不得隔页。


    \clearpage
